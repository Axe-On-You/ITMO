\thispagestyle{empty} % нет номеров страниц
\documentclass[12pt,a4paper]{report}
\usepackage[T1,T2A]{fontenc}
\usepackage[utf8x]{inputenc}
\usepackage[10pt]{extsizes}
\usepackage{geometry}
\usepackage{xcolor}
\usepackage{tikz}
\usepackage{graphicx, caption}
\usepackage{wrapfig}
\usepackage{amssymb}
\usepackage{setspace}


\geometry{top=10mm}
\geometry{bottom=5mm}
\geometry{left=10mm}
\geometry{right=10mm}

% Делаем 3 "новых" цвета (точные оттенки, которые на страничке), вместо стандартных
\definecolor{myorange}{HTML}{d94d1a}
\definecolor{myyellow}{HTML}{f2b31a}
\definecolor{myblue}{HTML}{265999}
%%%%%%%%%%%%%%%%%%

% Делаем команды, чтобы не повторять многократно один и тот же код
% В даннном случае имеем 7 различных рисунков

% 1. Прямая линия
\newcommand{\myline}[3]{
	\begin{tikzpicture}
    \draw[#1, ultra thick] (0,0) -- (1.5,0);
    \node[above] at (0,0) {\tiny#2};
    \node[above] at (1.5,0) {\tiny#3};
	\end{tikzpicture}
}

% 2. Угол BDE
\newcommand{\BDE}{
	\hspace{-0.6em}
	\raisebox{-15 pt}{
	\begin{tikzpicture}
	\filldraw[myyellow] (0,0) -- (0,0.7) arc (90:180:1.7 em);
	\filldraw[myorange] (0,0) -- (0,0.7) arc (90:60:1.7 em);
	\node[left] at (-0.7,0) {\tiny B};
	\node[below] at (0.1,0) {\tiny D};
	\node[above right] at (0.3,0.5) {\tiny E};
	\end{tikzpicture}
	}
	\hspace{-0.6em}
}

% 3. Угол CDE
\newcommand{\CDE}{
	\hspace{-0.6em}
	\raisebox{-15 pt}{
	\begin{tikzpicture}
	\filldraw[myblue] (0,0) -- (0.7,0) arc (0:60:1.7 em);
	\node[right] at (0.7,0) {\tiny C};
	\node[below] at (-0.1,0) {\tiny D};
	\node[above right] at (0.3,0.5) {\tiny E};
	\end{tikzpicture}
	}
	\hspace{-0.6em}
}

% 4. Полуокружность
\newcommand{\semicircle}{
	\raisebox{-5 pt}{
	\begin{tikzpicture}
	\draw[line width=1.5pt] (-0.7,0.7) -- (-0.7,0);
	\draw[line width=3pt] (90:0) arc (0:180:0.7);
	\draw[line width=1.5pt] (0.06,0) -- (-1.46,0);
	\end{tikzpicture}
	}
	\hspace{-0.6em}
}

% 5. Угол ABD
\newcommand{\ABD}{
	\hspace{-0.6em}
	\raisebox{-15 pt}{
	\begin{tikzpicture}
	\filldraw[myyellow] (0,0) -- (0,0.7) arc (90:180:1.7 em);
	\node[left] at (-0.7,0) {\tiny B};
	\node[below right] at (-0.05,0.1) {\tiny D};
	\node[above] at (0,0.65) {\tiny A};
	\end{tikzpicture}
	}
	\hspace{-0.6em}
}

% 6. Угол ACD
\newcommand{\ACD}{
	\hspace{-0.6em}
	\raisebox{-15 pt}{
	\begin{tikzpicture}
	\filldraw[myblue] (0,0) -- (0.7,0) arc (0:60:1.7 em);
	\filldraw[myorange] (0,0) -- (0,0.7) arc (90:60:1.7 em);
	\node[right] at (0.7,0) {\tiny C};
	\node[below] at (-0.15,0.1) {\tiny D};
	\node[above] at (0,0.65) {\tiny A};
	\end{tikzpicture}
	}
	\hspace{-0.6em}
}

% 7. Угол ADE
\newcommand{\ADE}{
	\hspace{-0.6em}
	\raisebox{-15 pt}{
	\begin{tikzpicture}
	\filldraw[myorange] (0,0) -- (0,0.7) arc (90:60:1.7 em);
	\node[above right] at (0.3,0.5) {\tiny E};
	\node[below] at (-0.15,0.1) {\tiny D};
	\node[above] at (0,0.65) {\tiny A};
	\end{tikzpicture}
	}
	\hspace{-0.6em}
}
%%%%%%%%%%%%%%%%%%%%%
% Костыль, чтобы предпоследняя сторчка уместилась в одну (без уменьшения её размера)
\newcommand{\plus}{
	\hspace{-0.6em}$+$\hspace{-0.6em}
}

\newcommand{\equal}{
	\hspace{-0.6em}$=$\hspace{-0.6em}
}

%%%%%%%%%%%%%%%%%%%%%

\begin{document}

\begin{minipage}{0.59\textwidth}
\hfill КНИГА I ПРЕДЛ. XIII. ТЕОРЕМА \hfill {\raisebox{-0.2em}{37}}
\vspace{3mm}
    \begin{wrapfigure}[6]{l}{0.19\textwidth}
        \includegraphics[width=0.2\textwidth]{img/E}
    \end{wrapfigure}

	сли \textit{прямая линия}
	\myline{myyellow}{E}{D}
	\textit{восставленная на другой прямой линии}
	\myline{myorange}{B}{C}
	\textit{образует с ней углы, то это будут либо два прямых
	угла, либо их сумма будет равна двум пря-мым углам.}
	
	\begin{center}

	Если\myline{myyellow}{E}{D}$\perp$ к\myline{myorange}{B}{C}тогда,\\
	\BDE \hspace{-0.6em}и\hspace{-0.6em}\CDE \equal \semicircle(опр. {\scriptsize IO}),\\
	но если\myline{myyellow}{E}{D}будет не $\perp$ к\myline{myorange}{B}{C},\\
	проведем \myline{black}{A}{D}$\perp$\myline{myorange}{B}{C}(пр. I.{\scriptsize II});\\
	\ABD \plus \ACD \equal \semicircle(постр.),\\
	\ABD \equal \ACD \equal \ADE \plus \CDE\\
	$\therefore$\ABD \plus \ACD \equal \ABD \plus \ADE \plus \CDE (акс. II)\\
	\equal \BDE \plus \CDE \equal \semicircle.

	\end{center}

	\begin{flushright}

	ч. т. д.

	\end{flushright}

\end{minipage}
\hfill
\begin{minipage}{0.35\textwidth}

	\begin{tikzpicture}[scale=1.2, line width=2pt]
		\filldraw[myyellow] (2.5,8) -- (2.5,8.5) arc (90:180:1.2 em);
		\filldraw[myorange] (2.5,8) -- (2.5,8.5) arc (90:60:1.2 em);
		\filldraw[myblue] (2.5,8) -- (3,8) arc (0:60:1.2 em);
		\draw[black] (2.5,8) -- (2.5,11);
		\draw[myyellow] (2.5,8) -- (3.5, 10);
		\draw[myorange] (0,8) -- (5,8);
		\node[above] at (2.5,10.9) {\tiny A};
		\node[below] at (0,8.05) {\tiny B};
		\node[below] at (5,8.05) {\tiny C};
		\node[below] at (2.5,8.05) {\tiny D};
		\node[above right] at (3.35,9.9) {\tiny E};
		\draw[white] (0,1.5) -- (1.7,1.5); % magic
	\end{tikzpicture}

\end{minipage}

\end{document}